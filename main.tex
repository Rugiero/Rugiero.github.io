% N.B. one {latexonly} environment commented out so that its
% contents can be displayed in the HTML version of this template.
% Uncomment it for actual use!
%
% Use text editor to replace:
%
%       author   --- author's login name
%       thisdoc  --- document filename (as in thisdoc.tex, thisdoc.ps)
%       psiz     --- size of compressed PostScript file
%
%       Document_Date          --- current date
%       Document_Short_Title   --- header text for Postscript
%       Document_Long_Title    --- full document title
%       Author_Name            --- full author name
%       Author_City            --- Charlottesville, Socorro, etc.
%       Author_State           --- Virginia, New Mexico, etc.
%
%       (non-NRAO: also replace institute name/acronym and country?)
%

\documentclass{article}
\usepackage{html,makeidx,epsf}
\renewcommand{\bibname}{References}

%
% Add home page navigation button -- edit the URL!
%


\htmladdtonavigation{\htmladdnormallink
  {\htmladdimg{../jetscalecropped.png}}{https://www.dropbox.com/s/jem3l3jabcyex9s/Curriculum_Vitae_Ilari_Angervuori.pdf?dl=0}}

%
% define hyperlink URLs:
%

\def\linkedin{https://www.linkedin.com/in/ilari-angervuori-0a1358160/}
\def\soundcloud{https://soundcloud.com/ilari-angervuori}
\def\github{https://github.com/Rugiero}
\def\cv{}
\makeindex

\begin{document}

%
%  Page formatting for Postscript output
%

\title{
{\bf A glimpse to my mind}
}

\author
{
Ilari Angervuori\\
}

\date
{
{Updated 09.02.2021}\\
}

\begin{center}
  \htmladdnormallink{Linkedin}{\linkedin}\\
  \htmladdnormallink{Soundcloud}{\soundcloud}\\
  \htmladdnormallink{GitHub}{\github} \\
  \htmladdnormallink{CV}{\cv}
\end{center}

%\begin{latexonly}
\markright{Document_Short_Title}
\maketitle
% uncomment to run:
%\end{latexonly}

\tableofcontents

\pagebreak
\chapter{About me}
\section{Education}
\section{Research}

\chapter{Blog posts 2021}
Random thoughts between Sun and Earth and Beyond.
\section{February}
\subsection{Controlling your passwords with pass}
Pass is a nice unix style free and open source wallet for keeping your passwords safe. Here is a brief look how to set it up in Ubuntu.

\begin{itemize}
\item Install the application in the terminal \\
\begin{verbatim}
sudo apt install pass  
\end{verbatim}
\item Check for existing GPG keys \\
\begin{verbatim}
gpg --list-keys 
\end{verbatim}
\item If no keys were found generate a key pair \\
\begin{verbatim}
gpg --generate-key
\end{verbatim}
\item Copy the name of the key and initalize pass\\
\begin{verbatim}
pass init ABCDEFGHIJKLMNOPQRSTUV1234 
\end{verbatim}
  \item Now you can add password with \\
\begin{verbatim}
pass generate keyfolder/newkey 
\end{verbatim}
\end{itemize}

Connect pass to git so it is easy to keep track of changes with multiple machines.

\begin{itemize}
\item Export your public and private key with \\
  \begin{verbatim}
gpg --export --output public.key ABCDEFGHIJKLMNOPQRSTUV1234 
gpg --export-secret-key --output private.key ABCDEFGHIJKLMNOPQRSTUV1234,
  \end{verbatim}
  where ABCDEFGHIJKLMNOPQRSTUV1234 is your keyname.\\
\item Now we can initilize the Git reporisoritory with these keys. Move public.key and private.key through a safe channel to the computer you wish to use pass in. Import the keys \\
\begin{verbatim}
gpg --import public.key
gpg --import private.key
\end{verbatim}
\item Initalize yout git repository\\
\begin{verbatim}
pass git init 
pass git remote add origin git@repo.com:myname/pass-store
 \end{verbatim}
\item Get password data from the server 
\begin{verbatim}
pass git pull origin master --allow-unrelated-histories
\end{verbatim}
\item Commit\\
\begin{verbatim}
pass git commit -am ``First'' 
\end{verbatim}
\item Push and set upstream \\
\begin{verbatim}
git push --set-upstream origin master 
\end{verbatim}
\item Pass commits the changes automatically and from here on you can use the familiar git commands \\
\begin{verbatim}
pass git pull 
pass git push 
\end{verbatim}
\end{itemize}
Stay safe.

\vspace{2cm}
References:
  \bibitem{stochasticgeometry}
    https://www.passwordstore.org/



%
% optional post-title formatting for PostScript
%
\parindent0pt
\parskip2.5ex plus 0.5ex minus 0.5ex
