% Free LaTeX novel template from Typesetters.se.

\documentclass{tsnovel}

\begin{document}

\tsnovel{In Latvia}
        {Ivan Rugiero}
        {<year of publication>}
        {<isbn>}
        {<publisher>}
        {<city where printed>}

\chapter{Sieniriemua}

\first{O}n toukokuun neljäs, ja kesän 2020 ensimmäisiä lämpimiä päiviä, kun poukkoilen puunjuurien alustamaa kinttupolkua alas kaupunkipyörällä. Siis sellaisella Helsinkiläisen lähikaupan mainoksilla kuorrutetulla polkupyörällä, joita saa lainattua pitkin kaupunkia sijoitetuista telineistä noin kolmenkymmenen euron kausivuokraa vastaan. Tänä kesänä pyöräpalvelu oli laajentunut jo Viikkiin, jonka metsäisiä luontopolkuja poljin nyt ympäriinsä auringonsäteiden täyteisen sään kunniaksi. Ihmisiä oli liikkeellä paljon – tämä Suomessa ja maailmalla riehuvasta korona-viruksesta huolimatta, sillä alkukevään pahin leviämisevaihe oli jo ohi, ja tietty tottumus ja huolettomuus  sairauden läsnäolosta vallitsi Suomen yltiö-onnellista kansaa, jota ei piinattu ulkonaliikkumiskielloilla samaan tapaan kuin suurta osaa varsinkin eteläisemmän euroopan maista.

Polku kaartaa hieman oikealle. Pidän oikeaa jalkaa suoristettuna valmiina ottamaan vastaan, jos pyörä lipsahtaisi pois uralta töyssyisellä polulla. Vasemmalla farkkujen polvessa on saumasta saumaan repsottava reikä, jonka liepeistä lepattaa lankoja tuulessa. Flat cap-lätsän alta repsottaa koronaeristystä liikaa nähneet hiukset.

Rymistän metsäpolulta lenkkipolulle hiekkatielle, jossa ihmiset kuljeksivat eri suuntiin  kävellen, ja juosten, ja pyörällä – miten ikinä: yksin, kaksin, perheiden kanssa. Singahtaessani lenkkipolulle pyöräni kanssa vastaan tuleva pariskunta katsoo minua ja naurahtaa. Onneksi minulla on päässä pyöreät tummat aurinkolasit, joiden taakse piiloutua, sillä pupillini ovat luultavasti hieman laajentuneet:  mikroannokseksi tarkoitettu psilobiinisieni-annos osoittautui taas toivottua toimivammaksi. Mutta olin onnellinen. Oli kaunis päivä ja olin rakastunut. Tuntui kuin universumin palaset olisivat loksahtaneet yhteen; kuin jonkinlainen mahdollisuuksien potentiaali olisi käynyt toteen – kaikki tuntui määrätyltä; koko aiempi elämä oli ollut valmistautumista tähän tilanteeseen. Tuntui kauniilta, mutta myös pelottavalta. Kun äkillinen rakkaus lyö mieheen, se voi lyödä tuhannen voimalla.  Sithän sanotaan, että se vie jalat alta.






\section{Vappu}

Kymmenen ihmisen kokoontumisrajoitus tarkoitti, että vappu 2020 oli peruttu. Kyyhötin vappuaattona kotona ja tissuttelin olutta ja kuuntelin teknoa ja keskustelin Tinderissä erään lupaavan naisen kanssa. Tumma, kaunis, älykäs. Sovimme tapaamisen seuraavalle päivälle vappupäiväksi. Sain valita tapaamispaikan, ja ehdotin, että tavataan Mustikkamaalla vappupicnicin merkeissä – säätiedotus lupasi kuitenkin kaunista ja kohtalaisen lämmintä päivää, ja pidin Mustikkamaan kallioisesta ja merenläheisestä metsäisestä luonnosta. Kävin kaupunginosassa säännöllisesti kävelemässä, sillä asuin tuohon aikaan Vallilassa Mäkelankadulla. Talossa, joka sijaitsi aivan puu-Vallilan kupeessa neljäkymmentä-luvulla rakennetussa kerrostalossa, minkä toisessa kerroksesa asuin kaapin kokoisessa yksiössä, josta maksoin tottakai suhteellisesti naurettavan korkeaa vuokraa. Ja siis – muusta urheilusta viis – rakastan kävelemistä. Otan kuulokkeet korviin, kuuntelen radio-ohjelmia tai youtube-videoita tai äänikirjoja kännykästä – helposti menee pari tuntia ja kymmenen kilometria kulkiessani päämäärättömästi pitkin Helsingin alueita korvissa  filosofiaa, proosaa, päivän politiikkaa, historiaa. Mikä nyt milloinkin kiinnosti. Pidän nykyajasta, jossa kaikki maailman tallenteet liikkuvat joka paikkaan taskussa ja korvakuulokkeissa. Mutta toisaalta on liiankin helppoa kulkea laput korvilla pitkin poikin ottamatta täydellistä aistiyhteyttä ympäristöön. Vähintääkin keväisin muuttolintujen saapuessa otan kuulokkeet joskus korvilta ja istun hetkeksi kuuntelemaan lintujen ajatonta viserrystä. Haistelen ilmaa ja tunnustelen auringon lämpösäteitä. Se on terveellistä.

Seuraavana päivänä saavuin Kalasataman Mustikkamaalle vievälle sillalle pyörällä jo hyvissä ajoin kolmen aikaa iltapäivällä. Minulla oli kassissa mukana pari olutta ja pullo \textit{Love}-merkkistä skumppaa. Lievä krapula kutitteli takaraivossa, sillä edellisenä iltana olin vielä ottanut taksin kotoani kaverin kutsumana vappujuhliin Pohjois-Haagaan, ja ilta venyi ehkä neljään asti aamulla. Ei se mitään – tunsin krapulaisen itseni tällä kertaa itsevarmaksi, sillä minulla oli hyvä fiilis näistä treffeistä. Olimme olleet Alban kanssa jo pidempään yhteydessä Tinderin välityksellä varsin välittömissä merkeissä. Koronapandemian ilmestymisen vuoksi viestittely oli katkennut joiksikin kuukausiksi, mutta se se alkoi nyt huhtikuussa samasta kuin mihin se oli helmikuussa päättynyt. Netissä huomasimme jo että meitä yhdisti moni asia – puhuimme musiikista ja syvänmeren kalmareista ja sen sellaisesta.

Alballa oli peilaavat aurinkolasit päässä kun hän saapui pyörällä sillalle. Tulin häntä pyörätelineelle vastaan. Halata ei voinut, sillä koronapandemian vuoksi oli pidettävä etäisyyttä. Alba hymyili kauniisti ja minua hieman kutkutti. Kävelimme sillan yli Mustikkamaalle. Alba kertoi, että oli tullut Suomeen alunperin vaihto-oppilaaksi Kouvolaan teollisen suunnittelun alalla ja jäänyt sittemmin Suomeen töihin. Suomessa hän oli asunut jo seitsemän vuotta, mutta Helsinkiin hän oli muuttanut vastikään opiskellakseen koodausta.

Istuimme kallioille lähelle hiekkarantaa. Muutama muu porukka oli myös saapunut vappupäivän picnicille – oli mukavaa ettemme olleet ainoita ihmisiä paikalla. Sää oli kohtalaisen lämmin, ja aurinko paisteli välillä pilvien lomasta. Joimme viiniä ja olutta ja Alban tuomia mausteisia juustoja ja mansikoita. Keskustellessamme englanniksi ja tutustuasse öysimme vähitellen uusia kosketuspintoja elämissämme ja persoonissamme. Alballa oli todellakin kaunis hymy. Hänellä oli päällään hieman hippimäisillä kuvioilla koristeltu vihreänharmaa villakangastakki. Jalassa hänellä oli korkeavartiset nahkakengät. Oikeastaan koin  pelkoa Alban koko olemusta kohtaan. Hän oli eksoottisen espanjalaisen kaunottaren näköinen kastanjapuun värisine hiuksineen ja viherkeltaisine silmineen – ja myös terve itsevarmuus paistoi hänestä. Olen aina ollut heikoilla itsetuntoni kanssa, ja koitin vain skarpata parhaani mukaan. Onneksi meillä oli kohtalaisesti alkoholia mukana, jolla saa pahimmat turhat kohinat päästä sammutettua, mietin. En tuntenut kuitenkaan alemmuuden tuntoa vaan vilpitöntä mielenkiintoa tummaa kaunotarta kohtaan.

Kävelimme yhdessä Mustikkamaalta Sörnaisten kurviin, josta Alba lähti kotiin Ullanlinnaa kohti. Hymyilimme ja sovimme tapaavamme vielä – vieläkään emme koskeneet toisiamme, ei edes koronatervehdystä, jossa kyynärpäillä hipaistaan toisiamme. Jotenkin oikeastaan pidän tällaisesta etäisyyden pitämisestä, olen aina ollut hieman huono halailemaan tai edes kättelemään ihmisiä. Kai siinä on jotain Suomalaista mentaliteettia pohjalla, mutta minä luultavasti olen tässä ``Suomalaisuudessa'' aivan omaa luokkaani.

Kävin vielä kaupasta ostamassa pari kaljaa, ja menin Vallilan yksiööni kuuntelemaan Chopinia. Rauhallisin vappu ikinä oli nyt pulkassa, ja onneksi seuraava päivä oli vielä viikonloppu.












\end{document}
