% Free LaTeX novel template from Typesetters.se.

\documentclass{tsnovel}

\begin{document}

\tsnovel{In Latvia}
        {Ivan Rugiero}
        {<year of publication>}
        {<isbn>}
        {<publisher>}
        {<city where printed>}

\chapter{Sieniriemua}

\first{O}n toukokuun neljäs, ja kesän 2020 ensimmäisiä lämpimiä päiviä, kun poukkoilen puunjuurien alustamaa kinttupolkua alas kaupunkipyörällä. Siis sellaisella Helsinkiläisen lähikaupan mainoksilla kuorrutetulla polkupyörällä, joita saa lainattua pitkin kaupunkia sijoitetuista telineistä noin kolmenkymmenen euron kausivuokraa vastaan. Tänä kesänä pyöräpalvelu oli laajentunut jo Viikkiin, jonka metsäisiä luontopolkuja poljin nyt ympäriinsä auringonsäteiden täyteisen sään kunniaksi. Ihmisiä oli liikkeellä paljon – tämä Suomessa ja maailmalla riehuvasta korona-viruksesta huolimatta, sillä alkukevään pahin leviämisevaihe oli jo ohi, ja tietty tottumus ja huolettomuus  sairauden läsnäolosta vallitsi Suomen yltiö-onnellista kansaa, jota ei piinattu ulkonaliikkumiskielloilla samaan tapaan kuin suurta osaa varsinkin eteläisemmän euroopan maista.

Polku kaartaa hieman oikealle. Pidän oikeaa jalkaa suoristettuna valmiina ottamaan vastaan, jos pyörä lipsahtaisi pois uralta töyssyisellä polulla. Vasemmalla farkkujen polvessa on saumasta saumaan repsottava reikä, jonka liepeistä lepattaa lankoja tuulessa. Flat cap-lätsän alta repsottaa koronaeristystä liikaa nähneet hiukset.

Rymistän metsäpolulta lenkkipolulle hiekkatielle, jossa ihmiset kuljeksivat eri suuntiin  kävellen, ja juosten, ja pyörällä – miten ikinä: yksin, kaksin, perheiden kanssa. Singahtaessani lenkkipolulle pyöräni kanssa vastaan tuleva pariskunta katsoo minua ja naurahtaa. Onneksi minulla on päässä pyöreät tummat aurinkolasit, joiden taakse piiloutua, sillä pupillini ovat luultavasti hieman laajentuneet:  mikroannokseksi tarkoitettu psilobiinisieni-annos osoittautui taas toivottua toimivammaksi. Mutta olin onnellinen. Oli kaunis päivä ja olin rakastunut. Tuntui kuin universumin palaset olisivat loksahtaneet yhteen; kuin jonkinlainen mahdollisuuksien potentiaali olisi käynyt toteen – kaikki tuntui määrätyltä; koko aiempi elämä oli ollut valmistautumista tähän tilanteeseen. Tuntui kauniilta, mutta myös pelottavalta. Kun äkillinen rakkaus lyö mieheen, se voi lyödä tuhannen voimalla.  Arvaamattomasti jalat alta.



\chapter{Votkan riemua}
Olin juuri rakastunut tuhannen kerran.



\chapter{Second chapter}

hurdur

\end{document}
